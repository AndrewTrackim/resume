% The font could be set to Windows-specific Calibri by using the 'calibri' option
\documentclass[]{mcdowellcv}

% For mathematical symbols
\usepackage{amsmath}
\usepackage[fixed]{fontawesome5}
\usepackage[hidelinks]{hyperref}

% Set applicant's personal data for header
\name{Varun Ramani}
\contacts{
	(732) 672-5930 % \faIcon{phone} 
	\linebreak
	varun.ramani@gmail.com % \faIcon{envelope} 
	\linebreak
	https://varunramani.com % \faIcon{globe}
	\linebreak
	https://github.com/varun-ramani % \faIcon{github} 
	\linebreak
	https://linkedin.com/in/varun-ramani % \faIcon{linkedin} 
}

\begin{document}

	% Print the header
	\makeheader
	
	% Print the content
	\begin{cvsection}{Education}
		\begin{cvsubsection}{College Park, MD}{University of Maryland}{August 2020 -- May 2023}
			\begin{itemize}
				\item B.S. in Computer Science. Current GPA: 4.0 / 4.0
				\item Coursework: Object Oriented Programming I \& II; Calculus I \& II; Introduction to Computer Systems (C, UNIX, and MIPS Assembly Programming); Discrete Structures; Linear Algebra
			\end{itemize}
		\end{cvsubsection}
	\end{cvsection}

	\begin{cvsection}{Technical Skills}
		\begin{cvsubsection}{}{}{}
			\vspace{0.5em}
			\begin{itemize}
				\item \textbf{Fluent:} Python, Flask, Java, NodeJS, JavaScript, MongoDB
				\item \textbf{Some Experience:} Go, PostgreSQL, Flutter, React Native, C
			\end{itemize}		
		\end{cvsubsection}
	\end{cvsection}

	\begin{cvsection}{Projects and Awards}
		\begin{cvsubsection}{Maskif.ai}{Grand Prize, YHack 2020}{\href{https://github.com/varun-ramani/maskifai-server}{gh:varun-ramani/maskifai-server}}
			\begin{itemize}
				\item Collaborated with 3 peers to develop accessible computer vision-powered IoT product.
				\item Product helps businesses deal with anti-maskers during pandemic by intelligently 
				triggering connected smart lock when unmasked individual approaches door; 
				automatically unlocks door after they leave.
				\item Beat 42 competing teams for first place.
				\item Applied Python, Tensorflow, Flask, and Google Assistant SDK. 
			\end{itemize}
		\end{cvsubsection}
		\begin{cvsubsection}{SkySpeech}{2nd Place \& Best Qualcomm Dragonboard Hack, HackPHS 2018}{\href{https://github.com/varun-ramani/skyspeech}{gh:varun-ramani/skyspeech}}
			\vspace{0.8em}
			\begin{itemize}
				\item Worked with 1 peer to develop mobile app and networked hub to aid in search and rescue missions; 
				product enables communications even in the absence of internet or cellular connectivity.
				\item Competed among at least 50 teams.
				\item Applied Python, Flask, React Native, Bootstrap, and Qualcomm Dragonboard 410c.
			\end{itemize}
		\end{cvsubsection}
        \begin{cvsubsection}{Intellicity}{Top 30, PennApps 2019}{\href{https://github.com/varun-ramani/intellicity}{gh:varun-ramani/intellicity}}
			\begin{itemize}
				\item Collaborated with 3 peers to develop advanced mobile map application.
				\item Product uses crowdsourced information and computer vision to add rich, granular details
				to Google Maps; includes but is not limited to precise geolocation data for trash bins, bathrooms, 
				safety hazards, and parking spots. Helps people navigate unfamiliar
				places with absolute confidence, instantly finding anything they need. 
				\item Competed against 242 other teams.
				\item Applied Dart/Flutter, Python 3, MongoDB, and Flask.
			\end{itemize}
		\end{cvsubsection}
		\begin{cvsubsection}{ZConfer}{26 Stars On GitHub}{\href{https://github.com/varun-ramani/zconfer}{gh:varun-ramani/zconfer}}
			\begin{itemize}
				\item Worked independently to design, develop, and publish comprehensive configuration tool
				for the Z Shell, a widely used command prompt program for UNIX-like systems.
				\item Abstracts away the tedious and often difficult task of writing configuration files;
				saves experienced users colossal amounts of time, while simultaneously making the Z Shell far
				more approachable for newcomers.
				\item Written in vanilla Python 3, used no dependencies to maximize performance and ease 
				of installation.
			\end{itemize}
		\end{cvsubsection}
	\end{cvsection}

	\begin{cvsection}{Employment}
		\begin{cvsubsection}{Sensei (teacher)}{Code Ninjas Princeton}{March 2019 -- December 2019}
			\begin{itemize}
				\item Used game development courses in Scratch and JavaScript to 
				teach students aged 7--14 introductory programming concepts.
				\item Conceptualized, developed, and led multi-day workshop on building NLP chatbots in 
				Python using IBM Watson.
				\item Developed and led mini-workshop covering computer vision in Scratch using IBM Watson.
				\item Supervised, mentored, and guided groups of up to 25 students at a time as they 
				built their software.
			\end{itemize}
		\end{cvsubsection}
	\end{cvsection}
\end{document}
